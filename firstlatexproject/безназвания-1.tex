\documentclass[12pt,a4paper]{article}


\usepackage[a4paper,margin=1in, left=10mm, top=20mm, right=10mm, bottom=20mm, nohead, nofoot]{geometry}
\usepackage[T2A]{fontenc}
\usepackage[utf8]{inputenc}
\usepackage[english, russian]{babel}
\usepackage{amsmath,amsthm}
\newtheorem{theorem}{Theorem}
\newtheorem{lemma}{Лемма}
\newtheorem{proposition}{Утверждение}
\usepackage{bussproofs}
\pagenumbering{gobble}

\begin{document}

Клеточный автомат, выполняющий сложение двух чисел в двоичной системе. К примеру, конфигурацию  $\wedge1011\ast 10$ , он пребразует в $1101\ast$ .

Вначале найдем единицу из второго числа, которую будем "перетаскивать" в число слева.

$(0,0,\wedge)\to 0'$

$(1,0,\wedge)\to 0'$

$(0,1,\wedge)\to 1''$

$(1,1,\wedge)\to 1''$

$(0,0,0')\to 0'$

$(1,0,0')\to 0'$

Итак, единицу мы нашли:

$(1,1,0')\to 1''$

$(0,1,0')\to 1''$

$(0,1'',0')\to 1'''$

$(1,1'',0')\to 1'''$

$(0,1,1'')\to 1'$

$(1,1,1'')\to 1'$

$(0,0,1'')\to 0''$

$(1,0,1'')\to 0''$

$(0,0'',A)\to 0$  A=\{0,1,1',1'',0',1''',0'',...,*\} - любой символ, кроме $\wedge$

$(1,0'',A)\to 0$

Теперь перетащим ее влево, за $\ast$:

$(0,*,1')\to *'$

$(1,*,1')\to *'$

$(A,*,*')\to *'$

$(A,*',A)\to *$

Найдем теперь место, куда вставить единицу:

$(1,1,*')\to 1^v$
$(1,0,*')\to 1$

$(0,1,*')\to 1^v$
$(0,0,*')\to 1$


$(1,1,1^v)\to 1^v$

$(0,1,1^v)\to 1^v$

$(1,0,1^v)\to 1$

$(0,0,1^v)\to 1$

$(\wedge,1,1^v)\to 1^v$

$(\wedge,\wedge,1^v)\to 1$

$(1,1^v,1)\to 0$

$(0,1^v,1)\to 0$

$(1,1^v,0)\to 0$

$(0,1^v,0)\to 0$

Осталось убрать лишние звездочки:

$(*,*,\wedge)\to \wedge$

Когда мы искали единицу, которую нужно перенести влево, мы портили правое число, нужно восстановить его:

$(1''',0',0')\to 1^w$

$(1''',0',\wedge)\to 1^w$

$(1''',\wedge,\wedge)\to \wedge$

$(A,1''',A)\to 0$

$(0,1^w,1)\to 1$

$(0,1^w,1)\to 1$

$(1,1^w,1)\to 1$

$(0,1^w,0')\to 1$

$(1,1^w,0')\to 1$

$(1^w,0',0')\to 1^w$

$(1^w,0',\wedge)\to 1^w$

$(0,1^w,\wedge)\to 1$

$(1,1^w,\wedge)\to 1$

Конец программы.
\end{document}

